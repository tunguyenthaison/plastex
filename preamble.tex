% Add support for mathematical equations.
\usepackage{amsmath}




% Add support for mathematical proof environment.
\usepackage{amsthm}
\usepackage{amssymb}

% Generate dummy text for testing.
\usepackage{lipsum}

% Use the typographical ruleset for the English language.
\usepackage[english]{babel}

% Nicer tables.
\usepackage{booktabs}

% Add support for bold mathematical symbols: $\bm{X}$.
\usepackage{bm}

% Add support for links (and urls in references).
% Dont outline the links themselves.
\usepackage[hidelinks]{hyperref}
\hypersetup{
   colorlinks = true,
   linkcolor = {red},
   citecolor = {blue},
}

% Add support for the degree symbol.
\usepackage{gensymb}

% Add support font shape definitions to use the Euler script symbols in math mode, we only need it if not using Euler math font
\usepackage{eucal}

% Set root folder for images.
\usepackage{graphicx}
\graphicspath{ {./images/} }

% Set document margins.
\usepackage{geometry}
\geometry{verbose,tmargin=3cm,bmargin=2.5cm,lmargin=2.5cm,rmargin=2.5cm,headheight=2.5cm}

% Add support for breaking the figure with \FloatBarrier.
\usepackage[above,below,verbose,section]{placeins}

% Add option for showing labels for equations.
% \usepackage[notcite,notref]{showkeys}

% Customize the font. Palatino with smallcap support.
\usepackage[sc]{mathpazo}
\usepackage[T1]{fontenc}
\usepackage{microtype}
\linespread{1.25}
% \usepackage[utopia]{mathdesign}

% Add simple colorbox for theorems and lemmas (trouble with texlive2023basic, work on overleaf - 2024-01-12).
% \usepackage[most]{tcolorbox}

% Add simple box for minipage.
\usepackage{boxedminipage}

% Set numbering for Section 1., and Subsections 1.1, none after that.
\setcounter{secnumdepth}{2}
% \setcounter{tocdepth}{3}
% \setcounter{equation}{0}
















% Macros. 
\newcommand{\R}{\mathbb{R}}
\newcommand{\T}{\mathbb{T}}






\newtheoremstyle{note}% <name>
{10pt}% <Space above>
{3pt}% <Space below>
%{\upshape}% <Body font>
{\itshape}% <Body font>
%{\parindent}% <Indent amount>
{0pt}% <Indent amount>
{\bfseries}% <Theorem head font>
{.}% <Punctuation after theorem head>
{0.5em}% <Space after theorem headi>
{}% <Theorem head spec (can be left empty, meaning `normal')>

\theoremstyle{note}
\newtheorem{theorem}{Theorem}
\newtheorem{corollary}[theorem]{Corollary}
\newtheorem{proposition}[theorem]{Proposition}
\newtheorem{lemma}[theorem]{Lemma}
\newtheorem{rem}[theorem]{Remark}
\newtheorem{problem}[theorem]{Problem}
\newtheorem{exercise}{Exercise}[section]
\newtheorem{example}{Example}[section]
\newtheorem{assumption}{Assumption}[section]
\newtheorem{definition}[theorem]{Definition}
\newtheorem{conjecture}[theorem]{Conjecture}

